\documentclass[a4paper, 12pt]{article}
\begin{document}
\pagenumbering{gobble}
\clearpage
\thispagestyle{empty}

\section{Introduction}

The problem.  The media is important to determine what people know about the world.  How the media gets its information is important as a result.  When media firms are getting information from their reporters they are getting infomration from other news media.

Our goal is to develop a model of media trust that accounts for trust in the media and how different media firms learn from each other.  Our model allows us to develop a network of trust between different media firms that allows for information diffusion between media firms. 

We examine two novel datasets to investigate this model, the first is a dataset of tweets for a list of 3000 media firms in the United States from June to August of 2014.  Additionally, we consider a dataset of articles taken from the free archive ArXiv.  We build a trust network for both of these datasets based on a number of different measures of bias and use this to investigate our model.

We specify our model using a topic model comparable to the popular topic modeling approach, Latent Dirichlet Allocation.  We uses this to estimate the amount of information that different firms get from eachother and uses this as an estimate of trust.

We consider 6 predictions.  First, firms that share common beliefs are more likely to trust each other.  Second, firms that share commmon beliefs will communicate information faster.  Third, higher quality firms are more likely to be trusted by all firms.  Fourth, information that originates from multiple distance media sources is more likely to be accurate.  Fifth, neutral media sources are important in order to connect different biased groups.  Sixth and finally, 

The paper is grouped as follows.  Section 2 presents the model and prior literature.  Section 3 convers the datasets used.  Section 4 covers the implementation of the model.  Section 5 presents the conclusions of the model.  Finally, summarizes and concludes.

\end{document}
