%
%
\documentclass[a4paper]{article}

\usepackage[english]{babel}
\usepackage[utf8]{inputenc}
\usepackage{amsmath}
\usepackage{graphicx}
\usepackage[colorinlistoftodos]{todonotes}

\title{Social Learning in the Media: The Importance of Bias for Trust Between Media Firms}
\author{Joseph Leland Bybee}
\date{\today}

\begin{document}

\maketitle

\section{Introduction}

The problem.  The media is important to determine what people know about the world.  How the media gets its information is important as a result.  When media firms are getting information from their reporters they are getting information from other news media.

Our goal is to develop a model of media trust that accounts for trust in the media and how different media firms learn from each other.  Our model allows us to develop a network of trust between different media firms that allows for information diffusion between media firms. 

We examine two novel data sets to investigate this model, the first is a data set of tweets for a list of 3000 media firms in the United States from June to August of 2014.  Additionally, we consider a data set of articles taken from the free archive ArXiv.  We build a trust network for both of these data sets based on a number of different measures of bias and use this to investigate our model.

We specify our model using a topic model comparable to the popular topic modeling approach, Latent Dirichlet Allocation (LDA).  We uses this to estimate the amount of information that different firms get from each other and uses this as an estimate of trust.

We consider 6 predictions.  First, firms that share common beliefs are more likely to trust each other.  Second, firms that share common beliefs will communicate information faster.  Third, higher quality firms are more likely to be trusted by all firms.  Fourth, information that originates from multiple distance media sources is more likely to be accurate.  Fifth, neutral media sources are important in order to connect different biased groups.  Sixth and finally, 

The paper is grouped as follows.  Section 2 presents the prior literature.  Section 3 develops the model used throughout.  Section 4 covers the data sets used.  Section 5 covers the implementation of the model.  Section 6 presents the results.  Finally, Section 7 summarizes and concludes.

\section{Literature Review}

We draw heavily from Gentzkow and Shapiro's 2006 paper \emph{Media Bias and Reputation}\footnote{Gentzkow and Shapiro.  Media Bias and Reputation. 2006}.  Gentzkow and Shapiro develop a model where media firms bias their reporting to align with the beliefs of their market to maximize profits.  We generalize this model to a network of media firms, where each firm learns from the actions of other firms in the network.

To estimate media bias we consider a number of different measures drawn from several different sources.  Groseclose and Milyo develop a measure that looks at the citations made by different media firms and how these coincide with the citations made by members of Congress\footnote{Groseclose and Milyo. A Measure of Media Bias. 2005.}.  Similarlly, Gentzkow and Shapiro develop a measure of media slant that uses the similarity between words used by different media outlets and members of Congress\footnote{Gentzkow and Shapiro.  What Drives Media Slant?  Evidence from U.S. Daily Newspaper. 2006.}.  Finally, Chiang and Knight develop a model of media bias that looks at endorsements made by media firms\footnote{Chiang and Knight.  Media Bias and Influence: Evidence from Newspaper Endorsements. 2011.}.

The literature on slant in academia is more limited than in the media case, perhaps because the impact is less apparent.  Much of the literature focuses on divisions on citation networks, though some looking at text clustering including one example presented below.  Onder and Tervio look at the division of citation networks into "freshwater" and "saltwater" departments for economics.  They find that the division is very real and significant in a number of cases\footnote{Onder and Tervio.  Is Economics a House Divided?  Analysis of Citation Networks.  2014.}.

The key work we draw from in the social learning literature is Acemoglu et. al's 2011 paper \emph{Bayesian Learning in Social Networks}\footnote{Acemoglu, Dahleh, Lobel and Ozdaglar.  Bayesian Learning in Social Networks. 2011.}.  They develop a model of Bayesian social learning the is built around a stochastic network similar to teh one we develop here.  They derive a number of general conclusions that are significant to this type of model. Acemoglu et. al's work on social learning in endogenous social networks is also significant as it mirrors the network structure we develop here\footnote{Acemoglu, Bimpikis and Ozdaglar.  Dynamic of Information Exchange in Endogenous Social Networks.  2013.}.  Non Bayesian approaches have also been considered in the literature including Jadbabaie et. al's work\footnote{Jadbabaie, Molavi, Sandroni and Tahbaz-Salehi.  Non-Bayesian Social Learning.  2012.}.

There is a large literature on more general information diffusion.  Young provides a good overview of how the social learning methods popular in economics compare to those used in other fields\footnote{Young.  Innovation Diffusion in Heterogeneous Populations: Contagion, Social Influence and Social Learning.  2014.}.  Golub and Jackson study how homophily affects diffusion and learning in networks.  Their results are directly relevant to the work done here\footnote{Golub and Jackson.  How Homophily Affects Learning and Diffusion in Networks. 2009.}.
 Yang and Leskovec develop a general model of information diffusion in implicit networks.  This is significant here because in most cases the network structure for our model is implicit\footnote{Yang and Leskovec.  Modeling Information Diffusion in Implicit Networks.  2010.}.

The method we use to estimate our model has its roots in the topic modeling literature, in particular the idea of LDA, first proposed by Blei, Ng and Jordan\footnote{Blei, Ng, Jordan. Latent Dirichlet Allocation. 2003.}.  LDA is essentially a hierarchical model where it is assumed that each word in a document is drawn from some topic distribution that is drawn for each document.  Using this hierarchical model Blei, Ng and Jordan proposed a method to estimate the underlying topic distribution for a set of documents.  Since their seminal paper the literature has exploded with examples of topic models.  Blei provides a good survey of the methodology\footnote{Blei.  Probabilitic Topic Models. 2012}.

Two key variants that are important here are dynamic topic models and author topic models.  A combination of these ideas form the basis for the estimation procedure used in this paper.  Blei
and Lafferty introduce the idea of dynamic topic models to represent how topic may change over time\footnote{Blei and Lafferty.  Dynamic Topic Models. 2006}.  They assume that the underlying topic proportions are drawn from some multivariate normal distribution centered around the prior topic distribution.  While we don't use the same implementation here, the idea a dynamic topic is essential to our work as our estimates for the trust edges represent how much a news story influences the underlying topic distribution for other media firms.  Rosen-Zvi et. al present another important variant in the idea of author topic models\footnote{Rosen-Zvi, Griffiths, Steyvers, Smyth.  The Author-Topic Model for Authors and Documents. 2004.}.  Their model assumes that each author has some underlying preference for certain topics that can be used to cluster documents.  While we don't directly implement this procedure the idea of a biased view of topics tied to each author is important to our model.

A number of studies have used similar methods to investigate similar datasets to ours.  Hong and Davison investigate and author topic model in the Twitter context to determine how Twitter users cluster and share information\footnote{Hong and Davison. Empirical Study of Topic Modeling in Twitter. 2010.}.  Grant et. al develop an online approach to estimating topics in the Twitter search context\footnote{Grant, George, Jenneisch and Wilson. Online Topic Modeling for Real-time Twitter Search. 2011.}.  Zhao et. al use a topic model to compare Twitter media to traditional media sources\footnote{Zhao, Jiang, Weng, He, Lim, Yan and Li. 2011.}.  Finally, Blei and Lafferty use a correlated topic model to estimate correlation within the scientific community using an academic article data set similar to the one used here\footnote{Blei and Lafferty. A Correlated Topic Model of Science. 2007.}.

\end{document}
